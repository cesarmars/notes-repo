\documentclass{article}
\usepackage{amsmath}
\usepackage{listings}
\usepackage{enumitem}

\title{Number Representation Notes}
\author{CS Lecture Notes}
\date{August 2025}

\begin{document}

\maketitle

\section{Representation Scheme}
\textit{Representation Scheme:} A method that tells us how to store information (e.g., numbers, addresses). \\

\noindent \textbf{Example:} Number representation

\section{Number Representation}
\textit{Number Representation:} A set of rules for how numbers are interpreted and encoded. \\

\noindent \textbf{Example:} Base systems

\section{Base / Radix}
\textit{Base:} A number system that defines the number of unique digits used to represent an integer. \\

\subsection*{Examples of Common Base Systems}
\begin{enumerate}
  \item Binary (Base-2) = 0, 1
  \item Octal (Base-8) = 0, 1, 2, 3, 4, 5, 6, 7
  \item Decimal (Base-10) = 0, 1, 2, 3, 4, 5, 6, 7, 8, 9
  \item Hexadecimal (Base-16) = 0--9 and A--F, where A = 10, B = 11, ..., F = 15
\end{enumerate}

\noindent All of these are number systems. \\

\noindent We typically index digits starting from 0, with the rightmost digit (the one's place) being position 0.

\begin{center}
\begin{tabular}{c c c c c}
Number Value: & 1 & 3 & 2 & 9 \\
Place Value:  & thousand's & hundred's & ten's & one's \\
Digit Index:  & $10^3$ & $10^2$ & $10^1$ & $10^0$ \\
\end{tabular}
\end{center}


\begin{center}
\begin{tabular}{c c c c c}
Number Value: & 1 & 0 . & 2 & 9 \\
Place Value:  & ten's & one's & tenth's & hundredth's \\
Digit Index:  & $10^1$ & $10^0$ & $10^{-1}$ & $10^{-2}$ \\
\end{tabular}
\end{center}




\section{Decimal (Base-10)}
\textit{Decimal:} A positional number system that represents any integer. \\

\noindent Digits for base-10 system: 0--9 \\

\textbf{Example:}
\[
12 = (1 \times 10^1) + (2 \times 10^0) = 12
\]

We use powers of 10 because there are 10 distinct digits. After 9, we roll over to the next place value:  
\[
9 \to 10, \quad 10 \times 10 = 100
\]

\section{Binary (Base-2)}
\textit{Binary:} A positional number system that represents two digits. \\

\noindent Digits: 0 and 1. \\

\noindent Binary prefixes: \texttt{0b}, subscript 2, or just binary digits. \\

\textbf{Example:}
\[
0b0100 = (0 \times 2^3) + (1 \times 2^2) + (0 \times 2^1) + (0 \times 2^0) = 4
\]

Binary rolls over after 1, just like decimal rolls over after 9.  
\[
\text{Decimal 2 = 0b0010, Decimal 4 = 0b0100}
\]

\noindent A \textbf{bit} is a binary digit (0 or 1). A single bit can represent anything, e.g., 0 = False, 1 = True. \\

\textbf{Example:}  
The binary \texttt{0b1010} has 4 bits, it is just the total number of 0's and 1's.

\section{More on Bits}
Common groupings:
\begin{itemize}
  \item Nibble = 4 bits
  \item Byte = 8 bits
  \item 2 Bytes = 16 bits
  \item 4 Bytes = 32 bits
\end{itemize}

\subsection*{Formulas}
\[
2^n = \text{Total number of values representable with $n$ bits}
\]
\[
2^n - 1 = \text{Maximum decimal value with $n$ bits}
\]
\[
2^{(n-1)} = \text{Value of the leftmost (most significant) bit}
\] \\

\textbf{Example:} With 4 bits:  
\[
2^4 = 16 \quad \Rightarrow \quad \text{16 values total}
\]
\[
2^4 - 1 = 15 \quad \Rightarrow \quad \text{Max decimal value = 15}
\]
\[
2^{(4-1)} = 2^3 = 8 \quad \Rightarrow \quad \text{Significant bit value = 8}
\]
\section{Examples of Binary and Decimal}

\subsection*{Binary to Decimal}
\textbf{Example 1:}

\texttt{0b11001011}

\begin{center}
\begin{tabular}{c c c c c c c c c}
Binary Value: & 1 & 1 & 0 & 0 & 1 & 0 & 1 & 1 \\
Digit Index:  & $2^7$ & $2^6$ & $2^5$ & $2^4$ & $2^3$ & $2^2$ & $2^1$ & $2^0$ \\
\end{tabular}
\end{center} 

\noindent Thus, \texttt{0b11001011}
\[
(1 \times 128) + (1 \times 64) + (0 \times 32) + (0 \times 16) + (1 \times 8) + (0 \times 4) + (1 \times 2) + (1 \times 1)
\]
\[
= 128 + 64 + 8 + 2 + 1 = 203
\]


\noindent \textbf{Example 2:}

\texttt{0b00111111}

\begin{center}
\begin{tabular}{c c c c c c c c c}
Binary Value: & 0 & 0 & 1 & 1 & 1 & 1 & 1 & 1 \\
Digit Index:  & $2^7$ & $2^6$ & $2^5$ & $2^4$ & $2^3$ & $2^2$ & $2^1$ & $2^0$ \\
\end{tabular}
\end{center} 

\noindent Thus, \texttt{0b00111111}:
\[
(0 \times 128) + (0 \times 64) + (1 \times 32) + (1 \times 16) + (1 \times 8) + (1 \times 4) + (1 \times 2) + (1 \times 1)
\]
\[
= 32 + 16 + 8 + 4 + 2 + 1 = 63
\]

\subsection*{Decimal to Binary}
\textbf{Example 1:}

\texttt{143, using 8 bits, our leftmost digit is 128 which is $2^{8-1}$}

\begin{center}
\begin{tabular}{c c c c c c c c c}
Digit Index: & 128 & 64 & 32 & 16 & 8 & 4 & 2 & 1 \\
Binary Value: & 1 & 0 & 0 & 0 & 1 & 1 & 1 & 1 \\
\end{tabular}
\end{center} 

\noindent Thus, \texttt{143} = 0b10001111 \\


\noindent \textbf{Example 2:} 

\texttt{9, using 4 bits, our leftmost digit is 8 which is $2^{4-1}$}

\begin{center}
\begin{tabular}{c c c c c}
Digit Index: & 8 & 4 & 2 & 1 \\
Binary Value: & 1 & 0 & 0 & 1 \\
\end{tabular}
\end{center}

\noindent Thus, 9  = 0b1001 \\

\noindent \textbf{Example 3:}

\texttt{3.14, using 4 integer bits for the decimal 3, and 4 fractional bits for .14 decimal } \\

\texttt{3}

\begin{center}
\begin{tabular}{c c c c c}
Digit Index: & 8 & 4 & 2 & 1 \\
Binary Value: & 0 & 0 & 1 & 1 \\
\end{tabular}
\end{center}

\noindent Thus, 3 = 0b0011 \\

\texttt{.14, for fractions, multiply by 2}

\begin{center}
\begin{tabular}{c c c c c}
fraction: & .14 & .28 & .56 & .12 \\
$fraction * 2$: & .28 & .56 & 1.12 & .24 \\
Binary: & 0 & 0 & 1 & 0 \\
\end{tabular}
\end{center}

\noindent So, the binary of $.14$ using 4 bits fractional is 
\[
.14 \approx 0b0.0010.
\]
We put a 1 at the third step because $0.56 \times 2 = 1.12 \geq 1$.  
That means the $2^{-3}$ place (the 1/8 place) is filled with a 1.  
The earlier steps gave 0s in the $2^{-1}$ and $2^{-2}$ places. \\

\noindent Thus, the binary of $3.14$ using 4 bits and 4 fractional bits is 
\[
3.14 \approx 0b0011.0010. 
\]

\noindent \textbf{Example 4:}

\texttt{.36 } \\

\begin{center}
\begin{tabular}{c c c c c}
fraction: & .36 & .72 & .44 & .88 \\
$fraction * 2$: & .72 & 1.44 & .88 & 1.76 \\
Binary: & 0 & 1 & 0 & 1 \\
\end{tabular}
\end{center}

\noindent So,
\[
.36 \approx 0b0.0101.
\]

\section{Hexadecimal (Base-16)}
\textit{Hexadecimal:} A positional number system that includes 16 digits. \\

\noindent Digits: 0-9, and A-F, where A = 10, B = 11, C = 12, D = 13, E = 14, F = 15. \\

\noindent Prefix: 0x \\

\textbf{Example:} "0x14"
\[
0x14 = (1 * 16^1) + (4 * 16^0) = 20
\]
What we did is convert hexadecimal to decimal using the power of 16 because we have 16 different digits. 
Additionally, we know that $16^1$ = $2^4$ = $16$, and $16^0$ = $2^0$ = $1$ thus we can write,
\[
0x14 = (1 * 16^1) + (4 * 16^0)
\]

\[
0x14 = (0b0001) * 2^4 + (0b0100) * 2^0 (\textit{Converted the values 1 and 4 to binary})
\]

\[
0x14 = (0b0001) 0000 + 0b0100  (\textit{concatenate})
\]

\[
    0x14 = 0b0001 0100 (\textit{Binary})
\]

Now we converted hexadecimal to binary. We converted 1 and 4 to binary because $2^4$ is $16^1$, and 
each hex digit maps to a nibble, then we concatenate. \\

\noindent Steps from hexadecimal to decimal.

1. Multiply the individual values by 16 with its respective position.

2. Then add those values to obtain the decimal. \\

\noindent Steps from decimal to hexadecimal.

1. Convert the whole decimal to binary.

2. Then split the binary into nibbles, resulting to hex form. \\

\noindent Steps from hexadecimal to binary. 

1. Convert the individual values to binary.

2. Then concatenate both binary values to obtain the final binary. \\

\noindent Steps from binary to hexadecimal. 

1. Split the binary into two nibbles.

2. Then match it to the right digit. 

\section{Examples of Hexadecimal}
\subsection*{Hexadecimal to Decimal}
\textbf{Example 1:} $0x5A$

\[
0x5A = (5 * 16^1) + (10 * 16^0) = (5 * 16) + 10 = 90
\] 

\noindent \textbf{Example 2:} $0xF1$ 

\[
0xF1 = (15 * 16^1) + (1 * 16^0) = (15 * 16) + 1 = 241
\] 

\subsection*{Decimal to Hexadecimal}
\textbf{Example 1:} $21$ using 8 bits because we want two nibbles to represent a hexadecimal, so our leftmost
digit is $2^{8-1} = 2^7 = 128$.

\begin{center}
\begin{tabular}{c c c c c c c c c}
Digit Index: & 128 & 64 & 32 & 16 & 8 & 4 & 2 & 1 \\
Binary Value: & 0 & 0 & 0 & 1 & 0 & 1 & 0 & 1 \\
\end{tabular}
\end{center} 

So,

\[
23 = 0b0001 0101 = 0x15
\]

\noindent \textbf{Example 2:} $9$ 

\begin{center}
\begin{tabular}{c c c c c c c c c}
Digit Index: & 128 & 64 & 32 & 16 & 8 & 4 & 2 & 1 \\
Binary Value: & 0 & 0 & 0 & 0 & 1 & 0 & 0 & 1 \\
\end{tabular}
\end{center} 

So, 

\[
9 = 0b0000 1001 = 0x09
\]

\noindent \textbf{Example 3:} $233$

\begin{center}
\begin{tabular}{c c c c c c c c c}
Digit Index: & 128 & 64 & 32 & 16 & 8 & 4 & 2 & 1 \\
Binary Value: & 1 & 1 & 1 & 0 & 1 & 0 & 0 & 1 \\
\end{tabular}
\end{center} 

So,

\[
233 = 0b1110 1001 = 0xE9
\]

\subsection*{Hexadecimal to Binary}
\textbf{Example 1:} $0x45$

\[
0x45 = (4 * 16^1) + (5 * 16^0)
\]

\[
0x45 = (0b0100 * 2^4) + 0b0101
\]

\[
0x45 = (0b0100 * 0000) + 0b0101 = 0b0100 0101 
\]

\noindent \textbf{Example 2:} $0x6B$

\[
0x6B = (6 * 16^1) + (11 * 16^0)
\]

\[
0x6B = (0b0110 * 2^4) + 0b1011
\]

\[
0x6B = (0b0110 * 0000) + 0b1011 = 0b0110 1011
\]

\subsection*{Binary to Hexadecimal}
\textbf{Example 1:} $0b1100 1100$

\begin{center}
\begin{tabular}{c c c}
Binary in two nibbles: & 1100 & 1100 \\
Hexadecimal value: & C & C \\
\end{tabular}
\end{center}

So,

\[
0b11001100 = 0xCC
\]

\noindent \textbf{Example 2:} $0b1010 0111$

\begin{center}
\begin{tabular}{c c c}
Binary in two nibbles: & 1010 & 0111 \\
Hexadecimal value: & A & 7 \\
\end{tabular}
\end{center} 

So,

\[
0b1010 0111 = 0xA7
\]

\section{Octal (Base-8)}
\textit{Octal:} A positional number system that includes 8 digits. \\

\noindent Digits: 0, 1, 2, 3, 4, 5, 6 ,7. \\

\noindent Prefix: 0o. \\

\noindent \textbf{Example 1:} "0o12"

\[
0o12 = (1 * 8^1) + (2 * 8^0) = 8 + 2 = 10
\]

We converted octal to decimal. Similar to hexadecimal, if you want to go from decimal to octal, you follow
the same steps. \\

\noindent \textbf{Example 2:} $83$

\begin{center}
\begin{tabular}{c c c c c c c c c c}
<<<<<<< HEAD
Digit Index: & 256 & 128 & 64 & 32 & 16 & 8 & 4 & 2 & 1 \\ 
=======
Digit Index: & 256 & 128 & 64 & 32 & 16 & 8 & 4 & 2 & 1 \\
>>>>>>> 42ef6c801f3d9ef8469b2f0d04ed4195b1f514b8
Binary Value: & 0 & 0 & 1 & 0 & 1 & 0 & 0 & 1 & 1 \\
\end{tabular}
\end{center}

So, 1 octal = 3 bits, we can group the binary to 3 bits, because $8^1 = 2^3$, where n = 3, number 
of bits.
\[
83 = 0b001 010 011 = 0o123
\]

\noindent \textbf{Example 3:} $0b101101$ to octal form

\[
0b101101 = 0b101\quad 0b101 = 5\quad 5 = 0o55.
\]

<<<<<<< HEAD
\section{Ways to Represent Signed Integers to Binary}
\subsection*{Bias}
\textit{Bias:} A fixed number added to a value so that the signed value is represented as a unsigned binary value. \\

\noindent General formulas for bias

\[
Signed: value = stored(signed) - bias
\]

\[
Unsigned: stored = value(unsigned)  + bias
\]

\noindent Bias is stated as 

\[
Bias = 2^{n-1} - 1
\]

\noindent \textbf{Ranges:} We can find the ranges for a given bit size. \\

\noindent For example, a 4 bit representation value range is:

\[
Min = stored - bias = 0 - bias = 0 - (2^{n-1} - 1) = 0 - (2^{4-1} - 1) = 0 - 7 = -7
\]

\[
Max = value + bias = (2^{n} - 1) + (-2^{n-1} - 1) = (2^{4} - 1) + (-7) = 15 - 7 = 8
\]

\noindent So the range is $[-7,8]$. \\

\noindent For a 8 bit representation value range is:

\[
Min = stored - bias = 0 - bias = 0 - (2^{n-1} - 1) = 0 - (2^{8-1} - 1) = 0 - 127 = -127
\]

\[
Max = value + bias = (2^{n} - 1) + (-2^{n-1} - 1) = (2^{8} - 1) + (-127) = 255 - 127 = 128
\]

\noindent So the range is $[-127,128]$ or in short $[-(2^{n-1} - 1), 2^{n-1}]$. \\

\subsection*{Examples of Bias}

\textbf{Example 1:} We have a 8 bit number 0b0000 1001 = 9. What is the stored value of the value 9. \\

This a unsigned integer so, 

\[
stored = value + bias = 9 - 127 = -118
\]

So what we did was convert a unsigned binary to a signed integer. Which is not the goal of a bias representation. The 
objective is to translate a signed integer into a unsigned binary representation. \\

\noindent \textbf{Example 2:} Store the value -9 to a unsigned binary representation using 8 bits. \\ 

This a signed integer so, 

\[
value = stored - bias = -9 -(-127) = 118 = 0b0111 0100
\]

So we have successfully converted a signed integer to a unsigned binary represenation, so $-9 = 0b0111 0100$ using bias 
representation. \\

\noindent In summary, bias representation turns signed integers into unsigned binary representation, and turns unsigned 
integers into signed values.

\subsection*{Two's Complement} 
\textit{Two's Complement:} A method of representing signed integer to unsigned binary representation, using two
distinct steps. This is how computers store sign integers. \\

\noindent \textbf{Ranges:} The range for two's complement is stated as,

\[
[-(2^{n-1}), 2^{n-1} - 1]
\]

\noindent Where $-2^{n-1}$ tells us the most negative number(aka MSB sign), and $2^{n-1} - 1$ tells us the most
positive number(aka magnitude). \\

\noindent The range of a 4 bit size: $[-(2^{n-1}), 2^{n-1} - 1] = [-2^{4-1}, 2^{4-1} - 1] = [-8, 7]$ \\

\noindent The range of a 8 bit size: $[-(2^{n-1}), 2^{n-1} - 1] = [-2^{8-1}, 2^{8-1} - 1] = [-128, 127]$ \\

\noindent \textbf{Steps for Two's Complement:} \\

1.) First, find the binary of the unsigned format of the signed integer. \\

2.) Second, get the one's complement of each bit, meaning $1 -> 0$, and $0 -> 1$. \\

3.) Finally, get the two's complement by adding one to that binary. \\

\noindent \textbf{Examples of Two's Complement} \\

\noindent \textbf{Example 1:} Convert -12 using the two's complement. \\

First, we know that -12 falls between a 8 bit size, $[-128, 127]]$:

\[
12 = 0b0000 1100
\] 

Second,

\[
0b0000 1100 = 0b1111 0011
\]

Finally,

\[
0b1111 0011 + 0b0000 0001 = 0b1111 0100
\]

Proof, our MSB is our sign bit, where $1 = -$ and $0 = +$.

\[
0b1111 0100 = -128 + 64 + 32 + 16 + 4 = -12
\]

\noindent \textbf{Example 2:} Convert -6 using the two's complement. \\

\[
6 = 0b0110
\]

\[
0b0110 = 0b1001
\]

\[
0b1001 + 0b0001 = 0b1010 = -8 + 2 = -6
\]

\subsection*{Sign Magnitude}
\textit{Sign Magnitude:} A method of representing signed integers as unsigned binary. \\

\noindent \textbf{Ranges:} The is range is given as, 

\[
[-(2^{n-1} - 1), 2^{n-1} - 1]
\]

\noindent The range for a 4 bit size: $[-(2^{4-1} - 1), 2^{4-1} - 1] = [-7, 7]$ \\

\noindent The range for a 8 bit size: $[-(2^{8-1} - 1), 2^{8-1} - 1] = [-127, 127]$ \\

\noindent \textbf{Example:} Represent $-7$ using sign magnitude. \\

We know that the signed integer falls between the range of a 4 bit size representation. So we can use 4 bits. \\
We always dedicate a bit the sign and the rest is our magnitude that make up the value 7. \\

Signs

\[
- = 0b1
\]

\[
+ = 0b0
\]

Magnitude of 7 using 3 bits

\[
7 = 0b111 
\]

Combine both the sign bit and magnitude

\[
-7 = 0b1111
\]

\noindent However, this has a lot of shortcomings like the lack of arthmetic logic in computing, it introduces a lot of
zero representations.

\section{Mathematical Operations on Binary}
\subsection*{Addition and Subtraction}
\noindent General rule for addition is. \\

$1 + 1 = 0$ means 0, you carry over a 1 to the next place
\[
1 + 1 = 0 
\]

\[
1 + 0 = 1
\]

\[
0 + 1 = 1
\]

\[
0 + 0 = 0
\]

\noindent \textbf{Example:} Involving 4 bit Two's Complement: $0b1001 + 0b0011$ \\

First check the range if the operation is possible, we know that the range for a 
bit in two's complement is [-8,7].
Where $0b1001 = 9$ and $0b0011 = 3$, $9$ is out of scope so we can not do this
operation. \\

\noindent \textbf{Example 2:} Involving 8 bit Two's Complement: $0b0000 1001
+ 0b0000 0011$ \\

The range of 8 bit: $[-127,128]$, both integers $9$ and $3$ falls between the 
range, so

\[
0b0000 1001 + 0b0000 0011 = 0b0000 1100
\]

\noindent \textbf{Example 3:} Involving 6 bit Two's Complement: $0b011 000 - 
0b000 011$ \\

The range of 6 bits: $[-32,31]$, both integers $24$ and $3$ fall in that range.
First lets find the two's complement of $-3$. 

\[
-3 = 3 = 0b000 011 = 0b111 100 = 0b111 100 + 1 = 0b111 101
\]

Then

\[
0b011 000 + 0b111 101 = 0b010101
\]

\noindent \textbf{Example 4:} Involving 6 bits Two's Complement: $0x3B - 0x06$ \\

Both values falls between the range. Lets find the two's complment of -6. Lets find the binary of -6 using two's 
complement.\\

\[
-6 = 6 = 0b000 110 = 0b111 001 = 0b111 001 + 1 = 0b111 010
\]

Then 

\[
0x3B - 0x06 = 0b11 1011 + 0b111 010 = 0b110 101
\]

\subsection*{Multiple} 
\noindent General rule for multiplying binaries,

\[
1 * 1 = 1
\]

\[
1 * 0 = 0
\]

\[
0 * 1 = 0
\]

\[
0 * 0 = 0
\]

\noindent \textbf{Example:} $0b001001 * 0b000011$, it is very similar to multiplying real numbers

\[
0b001001 * 0b000011 = 0b011011
\]




\end{document}

